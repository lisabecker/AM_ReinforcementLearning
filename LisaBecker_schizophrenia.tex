\documentclass[11pt,a4paper]{article}
\usepackage[utf8]{inputenc}
\usepackage[round]{natbib}
%\usepackage[style=authoryear,sorting=ynt]{biblatex}
\bibliographystyle{plainnat}
\usepackage[hyperref]{acl2018}
\usepackage{times}
\usepackage{latexsym}
\usepackage{url}
\usepackage{array}
\usepackage{graphicx}
\usepackage[font=small,labelfont=bf]{caption}
\graphicspath{ {./img/} } 
\pagestyle{plain}

\title{{\LARGE Reinforcement Learning for Natural Language Processing}\\[1.5mm]
{\large Final paper: Literature Review Article}\\[1.5mm]} %%% Fill in your project title
\author{Author: Lisa Becker (775242) } %%% Fill in your name 

\begin{document}
\maketitle

\section{Abstract}


\section{Introduction}
In 2016, DeepMind put Reinforcement Learning (RL) in the spotlights. Since then, RL has made many advantages in various subfields, one of them being Natural Language Processing (NLP). As stated in \citet{ijcai2019}, both domains influence each other. While we can gain new linguistic knowledge by getting insight into how RL agents deal with language, NLP can be used to enhance RL models. 

Knowledge representation

\section{Background}

\section{Current Use of NLP in RL}

\section{Trends for NLP in RL}

\section{Conclusion}

\newpage
\bibliography{bib}

\end{document}