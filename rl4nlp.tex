\documentclass[11pt,a4paper]{article}
\usepackage[utf8]{inputenc}
\usepackage[round]{natbib}
%\usepackage[style=authoryear,sorting=ynt]{biblatex}
\bibliographystyle{plainnat}
\usepackage[hyperref]{acl2018}
\usepackage{times}
\usepackage{latexsym}
\usepackage{url}
\usepackage{array}
\usepackage{graphicx}
\usepackage[font=small,labelfont=bf]{caption}
\graphicspath{ {./img/} } 
\pagestyle{plain}

\title{{\LARGE Reinforcement Learning for Natural Language Processing}\\[1.5mm]
{\large Final paper: Literature Review Article}\\[1.5mm]} 
\author{Author: Lisa Becker (775242) } 

\begin{document}
\maketitle
\section{Abstract}


\section{Introduction}
In 2016, DeepMind put Reinforcement Learning (RL) in the spotlights by developing AlphaGo \citep{alphago}. Since then, RL has made many advantages in various subdomains, one of them being Natural Language Processing (NLP). As stated in \citet{ijcai2019}, both domains influence each other. While we can gain new linguistic knowledge by getting insight into how RL agents deal with language, NLP can be used to enhance RL models. Since RL in NLP started to become more popular, various subdomains and applications emerged, such as Article summarization, Question Generation and Answering, Dialogue generation, Dialogue Systems, Machine Translation, Text generation.

\section{Background}

\section{Current Use of NLP in RL}


\section{Trends for NLP in RL}
Most RL models still utilise the REINFORCE algorithm or Deep Q Learning as their method.

\section{Problems for NLP in RL}
Research faces a few problems regarding the application of RL in NLP. 

The question of how important state is in RL from a NLP perspective has been discussed by \citet{madureira2020}.

\subsection{Data and Language}
While anglocentrism dominates most research fields, NLP and more specifically RL are no exception. This is due to English continuing to be the main language in which research is written and communicated and therefore most available corpora of sufficient size are English as well. While this is not a problem per se, it compels many researchers to also use English as their main language for conducting their experiments due to available data and funding for English research. This homogeneity 


\section{Conclusion}
As  \citet{ijcai2019} stated, "approaches combining language and RL will find applications as wide-ranging as autonomous vehicles, virtual assistants and household robots". While RL in NLP has made some advances, it is still far from being successfully exploited in Life Sciences. More careful research, especially in regard to the limitations and ethical implications, has to be conducted in order to model agents who can reliably applied to everyday lifes. 

\newpage
\bibliography{bib}

\end{document}