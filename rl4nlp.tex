\documentclass[11pt,a4paper]{article}
\usepackage[utf8]{inputenc}
\usepackage[round]{natbib}
%\usepackage[style=authoryear,sorting=ynt]{biblatex}
\bibliographystyle{plainnat}
\usepackage[hyperref]{acl2018}
\usepackage{times}
\usepackage{latexsym}
\usepackage{url}
\usepackage{array}
\usepackage{graphicx}
\usepackage[font=small,labelfont=bf]{caption}
\graphicspath{ {./img/} } 
\pagestyle{plain}

\title{{\LARGE Reinforcement Learning for Natural Language Processing}\\[1.5mm]
{\large Final paper: Literature Review Article}\\[1.5mm]} 
\author{Author: Lisa Becker (775242) } 

\begin{document}
\maketitle
\section{Abstract}


\section{Introduction}
In 2016, DeepMind put Reinforcement Learning (RL) in the spotlights. Since then, RL has made many advantages in various subfields, one of them being Natural Language Processing (NLP). As stated in \citet{ijcai2019}, both domains influence each other. While we can gain new linguistic knowledge by getting insight into how RL agents deal with language, NLP can be used to enhance RL models. 

-- Article summarization, Question Generation and Answering, Dialogue generation, Dialogue System, Knowledge-based QA, Machine Translation, Text generation

Knowledge representation

\section{Background}

\section{Current Use of NLP in RL}

\section{Trends for NLP in RL}

\section{Problems for NLP in RL}
Research faces a few problems regarding the application of RL in NLP. 

\subsection{Data and Language}
While anglocentrism dominates most research fields, NLP and more specifically RL are no exception. This is due to English continuing to be the main language in which research is written and communicated and therefore most available corpora of sufficient size are English as well. While this is not a problem per se, it compels many researchers to also use English as their main language for conducting their experiments. This leads to homogeneity 

\subsection{Reinforcement Learning}

\section{Conclusion}

\newpage
\bibliography{bib}

\end{document}